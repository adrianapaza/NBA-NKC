\title{Long views and Search Frustration: Revisiting NK models and the insights they provide about search and cognition }
% Use letters for affiliations, numbers to show equal authorship (if applicable) and to indicate the corresponding author
\author{
        Jon Atwell and Adrian Apaza}

\documentclass[12pt]{article}
\usepackage{amsmath}
\usepackage{graphicx}


\usepackage{apalike}
\begin{document}

\maketitle

\section*{Outline}

\subsection{Introduction}
  NK models are cool and all, but probably not a good starting point for understanding search processes because they likely induce a great amount of \textit{search frustration}--boredom or confusion with the process of adaptation. This is a feature of the particular way the NK model induces correlations within the landscape. (This still applies to Rivkin and Sigglekow's non-random interaction matrices, see \cite{Weinberger1990}. The focus of the model is ruggedness of the landscape but as a shaping parameter the number of couplings or epistatic interactions between components tunes more than the ruggedness. It reduces local correlations but at the expense of longer correlations. See again \cite{Weinberger1990}

\subsection*{The psychology of search}
  Humans are boundedly rational, but that does not mean they are ill-adapted psychologically for learning.

  Most search problems are rugged, but ruggedness is a local feature of the environment that human psychological impulses have learned to deal with.

  There are several reasons for this:

  i) search is stochastic because the feedback is not always reliable. Even the best betting strategy loses sometimes. We can deal with this.

  ii) Search in often nested and therefore strategic- the game within the game within the game - We know that winning battles doesn't win the war. We view information in light of larger strategy and potential outcomes.

  aside: is nested the same as hierarchical? No. Think of basketball. The unit of action is the game, but the goal is larger. Hierarchy implies you solve the game and then move onto the next level.

  iii)

So inspite of bounded rationality, humans are likely to be attuned to larger patterns in the adaptive landscape.

\subsection*{Decoupling Long and Short distance correlations: The Multiplicative NK model}

- Wait! What does this model do to local correlations? They must not change as fast as the NK so we're probably not really decoupling the two types of correlations. Is there a model with 2 tunable parameters to really decouple them? Something to think about

- Show the longer correlations are maintained even with local ruggedness

-  Point to appendix showing that basic NK results can differ

\subsection*{Experiment}

Now we show the difference in treatments for standard NK and the new one. 


\bibliography{/Users/jonathanatwell/Desktop/Library/library.bib}
\bibliographystyle{apalike}

\end{document}
